% ********** Rozdział 2 **********
\chapter{Instrukcja uruchomienia aplikacji}

Aby uruchomić projekt "Shakes \& Godmist" na swoim komputerze, należy kroki opisane w tym rozdziale.

\textbf{Uwaga:} Aplikacja została zaprojektowana z myślą o wdrożeniu na serwerze (np. VPS, chmura, serwer dedykowany). Uruchamianie lokalne służy wyłącznie celom deweloperskim i testowym. W środowisku produkcyjnym zaleca się wdrożenie backendu oraz bazy danych na serwerze, a frontend na serwerze WWW lub w usłudze hostingowej.

\section{Wymagania systemowe}
\begin{itemize}
    \item \textbf{System operacyjny:} Windows 10/11, Linux lub macOS
    \item \textbf{.NET SDK:} wersja 9.0 lub wyższa
    \item \textbf{Node.js:} wersja 18.x lub wyższa
    \item \textbf{Angular CLI:} wersja 20.x
    \item \textbf{PostgreSQL:} wersja 14.x lub wyższa
    \item \textbf{RAM:} minimum 4 GB (zalecane 8 GB)
    \item \textbf{Dysk:} minimum 500 MB wolnego miejsca
\end{itemize}

\section{Instalacja zależności}
\textbf{Backend (.NET):}
\begin{enumerate}
    \item Przejdź do katalogu backendu:
    \begin{verbatim}
    cd GameBackend
    \end{verbatim}
    \item Przywróć zależności NuGet:
    \begin{verbatim}
    dotnet restore
    \end{verbatim}
\end{enumerate}

\textbf{Frontend (Angular):}
\begin{enumerate}
    \item Przejdź do katalogu frontendu:
    \begin{verbatim}
    cd GameFrontend
    \end{verbatim}
    \item Zainstaluj zależności npm:
    \begin{verbatim}
    npm install
    \end{verbatim}
\end{enumerate}

\section{Konfiguracja bazy danych}
\begin{enumerate}
    \item Zainstaluj i uruchom serwer PostgreSQL.
    \item Utwórz nową bazę danych, np. o nazwie \texttt{shakesgodmist}.
    \item Skonfiguruj połączenie w pliku \texttt{appsettings.json} w katalogu GameBackend, np.:
    \begin{verbatim}
    "ConnectionStrings": {
      "DefaultConnection": "Host=localhost;Port=5432;
      Database=shakesgodmist;
      Username=postgres;Password=twoje_haslo"
    }
    \end{verbatim}
\end{enumerate}

\section{Uruchamianie backendu}
\begin{enumerate}
    \item Wykonaj migracje bazy danych:
    \begin{verbatim}
    dotnet ef database update
    \end{verbatim}
    \item Uruchom serwer backendu:
    \begin{verbatim}
    dotnet run
    \end{verbatim}
    \item Domyślnie backend będzie dostępny pod adresem: \url{http://localhost:5166}
\end{enumerate}

\section{Uruchamianie frontendu}
\begin{enumerate}
    \item W nowym terminalu przejdź do katalogu GameFrontend:
    \begin{verbatim}
    cd GameFrontend
    \end{verbatim}
    \item Uruchom serwer deweloperski Angular:
    \begin{verbatim}
    ng serve
    \end{verbatim}
    \item Frontend będzie dostępny pod adresem: \url{http://localhost:4200}
\end{enumerate}

\section{Dostęp do aplikacji}
\begin{itemize}
    \item \textbf{Panel użytkownika:} \url{http://localhost:4200}
    \item \textbf{API backendu:} \url{http://localhost:5166/api}
    \item \textbf{Swagger (dokumentacja API):} \url{http://localhost:5166/swagger}
\end{itemize}

\section{Najczęstsze problemy i wskazówki}
\begin{itemize}
    \item Jeśli port 5166 lub 4200 jest zajęty, zmień konfigurację w plikach projektu lub uruchom na innym porcie.
    \item W przypadku błędów połączenia z bazą danych sprawdź poprawność connection string i czy serwer PostgreSQL działa.
    \item Jeśli pojawią się błędy przy migracjach, upewnij się, że masz zainstalowane narzędzia Entity Framework Core CLI.
    \item W przypadku problemów z zależnościami npm uruchom \texttt{npm install} ponownie.
    \item W razie problemów z uprawnieniami uruchom terminal jako administrator.
\end{itemize}

% ********** Koniec rozdziału **********
